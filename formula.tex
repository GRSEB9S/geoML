
Kernel~Density~Estimation \\
\widehat{f}_N(x)=\frac{1}{Nh^D}\sum_{i=1}^NK\left(\frac{d(x,x_i)}{h}\right) \\
K(u)&is&the&kernel&function,&which&normalizes&to&unity.&(\int{K(u)du}=1) \\
h is known as the bandwidth. and mean is zero (\int{uK(u)du}=0) \\
and has a variance \sigma_k^2=\int{u^2}K(u)du greater than zero. \\
An oftern-used kernel is the Gaussian kernel,
K(u)=\frac{1}{(2\pi)^{D/2}}e^{-u^2/2}

Diffuse spectral reflectivity (I/F) is calculated by\\

IF_{x,\lambda}=\frac{pi{RD_{x,λ,r}}{SF_{x,λ} / r2 )

Subscripts include:
• x, spatial position in a row on the focal plane, in detector elements
• λ, wavelength in spectral direction on the focal plane, in detector elements or nm
• Hz, frame rate, wavelength table, and binning mode
• TaI, IR detector temperature in K for scene measurement
• TaV, VNIR detector temperature in K for scene measurement
• Ta2, spectrometer housing temperature in K
• Tc3, temperature of the integrating sphere in K for flight calibration measurement
• TaJ, IR focal plane board temperature in K for scene measurement
• TaW, VNIR focal plane board temperature in K for scene measurement
• t, integration time in seconds
• s, choice of sphere bulb, side 1 or 2

RD_{x,λ,r}&is&the&observed&spectral&radiance&in&W/(m^2 micrometer sr) at the instrument aperture.
SFx,λ is solar spectral irradiance for a normally illuminated surface 1 AU from the Sun,
convolved to the CRISM spectral profiles.
r is the distance of Mars from the Sun in astronomical units.
IFx,λ is the I/F of the scene (unitless), a measure of diffuse reflectivity.






